\documentclass[11pt]{amsart}
\usepackage{geometry}                % See geometry.pdf to learn the layout options. There are lots.
\geometry{letterpaper}                   % ... or a4paper or a5paper or ...
%\geometry{landscape}                % Activate for for rotated page geometry
%\usepackage[parfill]{parskip}    % Activate to begin paragraphs with an empty line rather than an indent
\usepackage{booktabs}
\usepackage{graphicx}
\usepackage{amssymb}
\usepackage{epstopdf}
\usepackage{caption}
\usepackage{subcaption}
\usepackage{commath}
\DeclareGraphicsRule{.tif}{png}{.png}{`convert #1 `dirname #1`/`basename #1 .tif`.png}

% Declare commands
\newcommand{\mat}[1]{\mathbf{#1}}

\title{CS 181 -- Practical 3}
\author{Casey Grun, Sam Kim, Rhed Shi}
%\date{}                                           % Activate to display a given date or no date

\begin{document}
\maketitle

% -----------------------------------------------------------------------------
\section{Warmup}


% -----------------------------------------------------------------------------
\section{Classification}

Our challenge was to classify a set of programs, based on traces of their system calls, as either a type of malware or a non-malicious program. Given an $N \times D$ matrix $\mat{X}$ representing the programs, along with a $N$-dimensional vector $\vec{t}$ of training labels, produce a function 
$$y : \mat{X'} \mapsto \vec{t'}$$.
That is, a function which could produce labels $\vec{t'}$ for some matrix $\mat{X'}$. There were two parts to this challenge: determining the \emph{feature functions} to generate rows of the matrix $\mat{X}$ based on the system calls for each malware program, and determining what classification algorithm to use to generate the function $y$.

\subsection{Feature Functions}

We evaluated a number of feature functions:
\begin{description}
  \item[System call counts]
  \item[$n$-grams]
  \item[DLLs loaded]
  \item[Registry keys]
\end{description}

\subsection{Classification Methods}



% -----------------------------------------------------------------------------
\section{Conclusion}


% -----------------------------------------------------------------------------
\begingroup
\begin{thebibliography}{9}

\bibitem{LSMR}
Fong, David Chin-Lung, and Michael Saunders. "LSMR: An iterative algorithm for sparse least-squares problems."
\emph{SIAM Journal on Scientific Computing} 33.5 (2011): 2950-2971

\end{thebibliography}
\endgroup

\end{document}